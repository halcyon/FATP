%*                                                                       *
%*                     Projet      Formel                                *
%*                                                                       *
%*                 Interface LaTeX -- MLgraph                            *
%*                                                                       *
%*************************************************************************
%*                                                                       *
%*                            LIENS                                      *
%*                        45 rue d'Ulm                                   *
%*                         75005 PARIS                                   *
%*                            France                                     *
%*                                                                       *
%*************************************************************************
%
%
% $Id: MLgraph.tex.p,v 1.4 1997/08/31 20:05:43 emmanuel Exp $
%*  MLgraph.tex  A interface between MLgraph and LaTeX                  *
%*               Ascander Suarez                                        *
%                Mon Apr 26 1993                                        *
%                                                                       *
%                modified by Emmanuel Chailloux                         *
%                    (thanks to Roberto Dicosmo for his help)           *
%                Wed Dec 18 1996
% Usage:
%   %*                                                                       *
%*                     Projet      Formel                                *
%*                                                                       *
%*                 Interface LaTeX -- MLgraph                            *
%*                                                                       *
%*************************************************************************
%*                                                                       *
%*                            LIENS                                      *
%*                        45 rue d'Ulm                                   *
%*                         75005 PARIS                                   *
%*                            France                                     *
%*                                                                       *
%*************************************************************************
%
%
% $Id: MLgraph.tex.p,v 1.4 1997/08/31 20:05:43 emmanuel Exp $
%*  MLgraph.tex  A interface between MLgraph and LaTeX                  *
%*               Ascander Suarez                                        *
%                Mon Apr 26 1993                                        *
%                                                                       *
%                modified by Emmanuel Chailloux                         *
%                    (thanks to Roberto Dicosmo for his help)           *
%                Wed Dec 18 1996
% Usage:
%   %*                                                                       *
%*                     Projet      Formel                                *
%*                                                                       *
%*                 Interface LaTeX -- MLgraph                            *
%*                                                                       *
%*************************************************************************
%*                                                                       *
%*                            LIENS                                      *
%*                        45 rue d'Ulm                                   *
%*                         75005 PARIS                                   *
%*                            France                                     *
%*                                                                       *
%*************************************************************************
%
%
% $Id: MLgraph.tex.p,v 1.4 1997/08/31 20:05:43 emmanuel Exp $
%*  MLgraph.tex  A interface between MLgraph and LaTeX                  *
%*               Ascander Suarez                                        *
%                Mon Apr 26 1993                                        *
%                                                                       *
%                modified by Emmanuel Chailloux                         *
%                    (thanks to Roberto Dicosmo for his help)           *
%                Wed Dec 18 1996
% Usage:
%   %*                                                                       *
%*                     Projet      Formel                                *
%*                                                                       *
%*                 Interface LaTeX -- MLgraph                            *
%*                                                                       *
%*************************************************************************
%*                                                                       *
%*                            LIENS                                      *
%*                        45 rue d'Ulm                                   *
%*                         75005 PARIS                                   *
%*                            France                                     *
%*                                                                       *
%*************************************************************************
%
%
% $Id: MLgraph.tex.p,v 1.4 1997/08/31 20:05:43 emmanuel Exp $
%*  MLgraph.tex  A interface between MLgraph and LaTeX                  *
%*               Ascander Suarez                                        *
%                Mon Apr 26 1993                                        *
%                                                                       *
%                modified by Emmanuel Chailloux                         *
%                    (thanks to Roberto Dicosmo for his help)           *
%                Wed Dec 18 1996
% Usage:
%   \input{MLgraph}
%   \begin{mlPic}
%      \mlOpen{fa}
%      \mlOpen{fb}
%      \mlDir{da}
%      \mlDir{db}
%      \mlDef{a}{First LaTeX text}
%      \mlDef{b}{Second LaTeX text}
%      etc.
%      \mlBody{p}{ 
%          Your MLgraph picture which might use latexBox"a",latexBox"b",etc.
%          of type MLgraph__picture
%      }
%      \end{mlPic}
%
%Supposes that your LaTeX file uses the TeX box declaration conventios
\newcount{\mlBoxCount}\newtoks{\mlToks}\newread{\mlRead}\newlength{\mlLen}
\makeatletter
\def\makeMlPicTex{\if@filesw \newwrite\@mlwritefile
  \immediate\openout\@mlwritefile=\jobname.pic/\jobname.ml
  \mlwrite%
  {(****************************************************************}%
  \mlwrite%
  { *                                                               }%
  \mlwrite%
  { * File \jobname.pic/\jobname.ml created by  MLgraph v2.1 from \jobname.tex   }%
  \mlwrite%
  { *                                                               }%
  \mlwrite%
  { ****************************************************************)}%
  \mlwrite{ }%
  \mlwrite{include "mlgraph.ml";;}
  \mlwrite{picDir:="\jobname.pic";;}
  \mlwrite{set_latex_mode ();;}
%  \mlwrite{let addLatexBox lb = latexBoxTable:=!latexBoxTable@[lb];;}
  \mlwrite{ }%
  \typeout
  {Writing MlPicTeX file \jobname.pic/\jobname.ml }\fi}
 
\def\@wrmlwrite#1{\let\thepage\relax
   \edef\@tempa{\write\@mlwritefile{\string #1}}\expandafter\endgroup\@tempa
   \if@nobreak \ifvmode\nobreak\fi\fi\@esphack}
  \def\mlwrite{\@bsphack\begingroup\@sanitize\@wrmlwrite}
\makeatother
% \mlDef{name}{text} defines a value (latexBox"name") of type MLgraph__picture
% which can be used as part of the resulting picture
\newcommand{\mlDef}[2]{% Declaration of horizontal boxes
    \advance\mlBoxCount by 1%
    \setbox\mlBoxCount=\hbox{#2}%
  \mlwrite%
  {(****************************************************************}%
  \mlwrite%
  { *  building box  : #1}%
  \mlwrite%
  { ****************************************************************)}%
    \mlwrite%
{add_latexBox (("#1",\the\mlBoxCount,"\the\wd\mlBoxCount","\the\ht\mlBoxCount","\the\dp\mlBoxCount"));;}}%%%%%%%%%%::!latexBoxTable;}}
\newcommand{\mlOpen}[1]{%
	\mlwrite{##open #1;;}%
%	\mlwrite{open #1;;}%
        \mlwrite{load_object #1;;}%
}%
\newcommand{\mlDir}[1]{%
	\mlwrite{\#}%
	\mlwrite{#1;;}%
}%
\newcommand{\mlDefGen}[3]{% Declaration of horizontal boxes
    \advance\mlBoxCount by 1%
    \setbox\mlBoxCount=\hbox{#3}%
    \mlwrite%
{latexBoxTable:=!latexBoxTable@[(latexPictureGen #2 ("#1",\the\mlBoxCount,"\the\wd\mlBoxCount","\the\ht\mlBoxCount","\the\dp\mlBoxCount"))];;}}%
\newcommand{\mlDefBox}[2]{% Inclusion of an existent box
    \mlwrite%
{latexBoxTable:=!latexBoxTable@[(latexPicture("#1",\the#2,"\the\wd#2","\the\ht#2","\the\dp#2"))];;}}%
\newcommand{\mlVDef}[2]{%  Declaration of vertical boxes
    \advance\mlBoxCount by 1%
    \setbox\mlBoxCount=\vbox{#2}%
    \mlwrite%
{latexBoxTable:=!latexBoxTable@[latexPicture("#1",\the\mlBoxCount,"\the\wd\mlBoxCount","\the\ht\mlBoxCount","\the\dp\mlBoxCount")];;}}%
% Expression of type MLgraph__picture that describes the picture
\newcommand{\mlBody}[2]{\mlToks={\jobname.pic/#1.tex}%
  \mlwrite{ }
  \mlwrite%
  {(****************************************************************}%
  \mlwrite%
  { * building picture #1 }
  \mlwrite%
  { ****************************************************************)}%
  \mlwrite{ }
%   \mlwrite{let latexBox s = fst(assoc s !latexBoxTable) }%
%   \mlwrite{and latexBoxes = map (fun (s,(p,_)) -> (s,p)) !latexBoxTable }%
%   \mlwrite{in }%
   \mlwrite{ let mlGraphPicture = }%
   \mlwrite{  #2 }%
   \mlwrite{  in makeLatexPicture mlGraphPicture !latexBoxTable "#1";; }%
}
% initializations for MLgraph pictures
\newenvironment{mlPic}{%
  \mlBoxCount=\the\count14% the box counter
   \mlwrite{(*--------- mlPic Description output ---------*)}%
   \mlwrite{ }%
  \mlwrite%
  {(****************************************************************}%
  \mlwrite%
  { * starting a new picture}%
  \mlwrite%
  { ****************************************************************)}%
  \mlwrite{ }%
   \mlwrite{empty_latexBoxTable ();;}%
   \mlwrite{ }%
 }{%
  \mlwrite{ }%
  \mlwrite{(*------End of mlPic description output ------*)}%
  \mlwrite{ }%
\mlReadFun{\the\mlToks}}
\newcommand{\mlReadFun}[1]{\openin\mlRead=#1
\ifeof\mlRead\closein\mlRead\message{#1\space is empty}%
\else{\closein\mlRead}\setlength{\mlLen}{\unitlength}%
\setlength{\unitlength}{1pt}\input{#1}%
\setlength{\unitlength}{\mlLen}\fi}
\newcommand{\mlPut}[3]{\put#1{\special{ps:gsave currentpoint currentpoint translate [ #2 0 0]concat neg exch neg exch translate}#3\special{ps:grestore}}}
\newcommand{\mlPutg}[4]{\put#1{\special{ps:gsave currentpoint currentpoint translate [ #2 0 0]concat neg
exch neg exch translate #3 setgray }#4\special{ps:grestore}}}
\newcommand{\mlPutrgb}[4]{\put#1{\special{ps:gsave (start) ==
[1 0 0 1 0 0 ] defaultmatrix ==
currentpoint [1 0 0 1 0 0] currentmatrix
initmatrix
currentpoint translate
[1 0 0 1 0 0] currentmatrix ==
#3
setrgbcolor  setmatrix moveto  currentpoint currentpoint translate [ #2 0 0]
concat neg exch neg exch translate  }#4\special{ps:grestore}}}
\newcommand{\mlPuthsb}[4]{\put#1{\special{ps:gsave currentpoint currentpoint translate [ #2 0 0]concat neg exch neg exch translate #3 sethsbcolor }#4\special{ps:grestore}}}
\newcommand{\mlPutgen}[4]{\put#1{\special{ps:gsave  currentpoint [1 0 0 1 0 0] currentmatrix initmatrix currentpoint translate #3 setmatrix moveto  currentpoint currentpoint translate [ #2 0 0]
concat neg exch neg exch translate}#4\special{ps:currentpoint grestore moveto}}}
\makeMlPicTex

%   \begin{mlPic}
%      \mlOpen{fa}
%      \mlOpen{fb}
%      \mlDir{da}
%      \mlDir{db}
%      \mlDef{a}{First LaTeX text}
%      \mlDef{b}{Second LaTeX text}
%      etc.
%      \mlBody{p}{ 
%          Your MLgraph picture which might use latexBox"a",latexBox"b",etc.
%          of type MLgraph__picture
%      }
%      \end{mlPic}
%
%Supposes that your LaTeX file uses the TeX box declaration conventios
\newcount{\mlBoxCount}\newtoks{\mlToks}\newread{\mlRead}\newlength{\mlLen}
\makeatletter
\def\makeMlPicTex{\if@filesw \newwrite\@mlwritefile
  \immediate\openout\@mlwritefile=\jobname.pic/\jobname.ml
  \mlwrite%
  {(****************************************************************}%
  \mlwrite%
  { *                                                               }%
  \mlwrite%
  { * File \jobname.pic/\jobname.ml created by  MLgraph v2.1 from \jobname.tex   }%
  \mlwrite%
  { *                                                               }%
  \mlwrite%
  { ****************************************************************)}%
  \mlwrite{ }%
  \mlwrite{include "mlgraph.ml";;}
  \mlwrite{picDir:="\jobname.pic";;}
  \mlwrite{set_latex_mode ();;}
%  \mlwrite{let addLatexBox lb = latexBoxTable:=!latexBoxTable@[lb];;}
  \mlwrite{ }%
  \typeout
  {Writing MlPicTeX file \jobname.pic/\jobname.ml }\fi}
 
\def\@wrmlwrite#1{\let\thepage\relax
   \edef\@tempa{\write\@mlwritefile{\string #1}}\expandafter\endgroup\@tempa
   \if@nobreak \ifvmode\nobreak\fi\fi\@esphack}
  \def\mlwrite{\@bsphack\begingroup\@sanitize\@wrmlwrite}
\makeatother
% \mlDef{name}{text} defines a value (latexBox"name") of type MLgraph__picture
% which can be used as part of the resulting picture
\newcommand{\mlDef}[2]{% Declaration of horizontal boxes
    \advance\mlBoxCount by 1%
    \setbox\mlBoxCount=\hbox{#2}%
  \mlwrite%
  {(****************************************************************}%
  \mlwrite%
  { *  building box  : #1}%
  \mlwrite%
  { ****************************************************************)}%
    \mlwrite%
{add_latexBox (("#1",\the\mlBoxCount,"\the\wd\mlBoxCount","\the\ht\mlBoxCount","\the\dp\mlBoxCount"));;}}%%%%%%%%%%::!latexBoxTable;}}
\newcommand{\mlOpen}[1]{%
	\mlwrite{##open #1;;}%
%	\mlwrite{open #1;;}%
        \mlwrite{load_object #1;;}%
}%
\newcommand{\mlDir}[1]{%
	\mlwrite{\#}%
	\mlwrite{#1;;}%
}%
\newcommand{\mlDefGen}[3]{% Declaration of horizontal boxes
    \advance\mlBoxCount by 1%
    \setbox\mlBoxCount=\hbox{#3}%
    \mlwrite%
{latexBoxTable:=!latexBoxTable@[(latexPictureGen #2 ("#1",\the\mlBoxCount,"\the\wd\mlBoxCount","\the\ht\mlBoxCount","\the\dp\mlBoxCount"))];;}}%
\newcommand{\mlDefBox}[2]{% Inclusion of an existent box
    \mlwrite%
{latexBoxTable:=!latexBoxTable@[(latexPicture("#1",\the#2,"\the\wd#2","\the\ht#2","\the\dp#2"))];;}}%
\newcommand{\mlVDef}[2]{%  Declaration of vertical boxes
    \advance\mlBoxCount by 1%
    \setbox\mlBoxCount=\vbox{#2}%
    \mlwrite%
{latexBoxTable:=!latexBoxTable@[latexPicture("#1",\the\mlBoxCount,"\the\wd\mlBoxCount","\the\ht\mlBoxCount","\the\dp\mlBoxCount")];;}}%
% Expression of type MLgraph__picture that describes the picture
\newcommand{\mlBody}[2]{\mlToks={\jobname.pic/#1.tex}%
  \mlwrite{ }
  \mlwrite%
  {(****************************************************************}%
  \mlwrite%
  { * building picture #1 }
  \mlwrite%
  { ****************************************************************)}%
  \mlwrite{ }
%   \mlwrite{let latexBox s = fst(assoc s !latexBoxTable) }%
%   \mlwrite{and latexBoxes = map (fun (s,(p,_)) -> (s,p)) !latexBoxTable }%
%   \mlwrite{in }%
   \mlwrite{ let mlGraphPicture = }%
   \mlwrite{  #2 }%
   \mlwrite{  in makeLatexPicture mlGraphPicture !latexBoxTable "#1";; }%
}
% initializations for MLgraph pictures
\newenvironment{mlPic}{%
  \mlBoxCount=\the\count14% the box counter
   \mlwrite{(*--------- mlPic Description output ---------*)}%
   \mlwrite{ }%
  \mlwrite%
  {(****************************************************************}%
  \mlwrite%
  { * starting a new picture}%
  \mlwrite%
  { ****************************************************************)}%
  \mlwrite{ }%
   \mlwrite{empty_latexBoxTable ();;}%
   \mlwrite{ }%
 }{%
  \mlwrite{ }%
  \mlwrite{(*------End of mlPic description output ------*)}%
  \mlwrite{ }%
\mlReadFun{\the\mlToks}}
\newcommand{\mlReadFun}[1]{\openin\mlRead=#1
\ifeof\mlRead\closein\mlRead\message{#1\space is empty}%
\else{\closein\mlRead}\setlength{\mlLen}{\unitlength}%
\setlength{\unitlength}{1pt}\input{#1}%
\setlength{\unitlength}{\mlLen}\fi}
\newcommand{\mlPut}[3]{\put#1{\special{ps:gsave currentpoint currentpoint translate [ #2 0 0]concat neg exch neg exch translate}#3\special{ps:grestore}}}
\newcommand{\mlPutg}[4]{\put#1{\special{ps:gsave currentpoint currentpoint translate [ #2 0 0]concat neg
exch neg exch translate #3 setgray }#4\special{ps:grestore}}}
\newcommand{\mlPutrgb}[4]{\put#1{\special{ps:gsave (start) ==
[1 0 0 1 0 0 ] defaultmatrix ==
currentpoint [1 0 0 1 0 0] currentmatrix
initmatrix
currentpoint translate
[1 0 0 1 0 0] currentmatrix ==
#3
setrgbcolor  setmatrix moveto  currentpoint currentpoint translate [ #2 0 0]
concat neg exch neg exch translate  }#4\special{ps:grestore}}}
\newcommand{\mlPuthsb}[4]{\put#1{\special{ps:gsave currentpoint currentpoint translate [ #2 0 0]concat neg exch neg exch translate #3 sethsbcolor }#4\special{ps:grestore}}}
\newcommand{\mlPutgen}[4]{\put#1{\special{ps:gsave  currentpoint [1 0 0 1 0 0] currentmatrix initmatrix currentpoint translate #3 setmatrix moveto  currentpoint currentpoint translate [ #2 0 0]
concat neg exch neg exch translate}#4\special{ps:currentpoint grestore moveto}}}
\makeMlPicTex

%   \begin{mlPic}
%      \mlOpen{fa}
%      \mlOpen{fb}
%      \mlDir{da}
%      \mlDir{db}
%      \mlDef{a}{First LaTeX text}
%      \mlDef{b}{Second LaTeX text}
%      etc.
%      \mlBody{p}{ 
%          Your MLgraph picture which might use latexBox"a",latexBox"b",etc.
%          of type MLgraph__picture
%      }
%      \end{mlPic}
%
%Supposes that your LaTeX file uses the TeX box declaration conventios
\newcount{\mlBoxCount}\newtoks{\mlToks}\newread{\mlRead}\newlength{\mlLen}
\makeatletter
\def\makeMlPicTex{\if@filesw \newwrite\@mlwritefile
  \immediate\openout\@mlwritefile=\jobname.pic/\jobname.ml
  \mlwrite%
  {(****************************************************************}%
  \mlwrite%
  { *                                                               }%
  \mlwrite%
  { * File \jobname.pic/\jobname.ml created by  MLgraph v2.1 from \jobname.tex   }%
  \mlwrite%
  { *                                                               }%
  \mlwrite%
  { ****************************************************************)}%
  \mlwrite{ }%
  \mlwrite{include "mlgraph.ml";;}
  \mlwrite{picDir:="\jobname.pic";;}
  \mlwrite{set_latex_mode ();;}
%  \mlwrite{let addLatexBox lb = latexBoxTable:=!latexBoxTable@[lb];;}
  \mlwrite{ }%
  \typeout
  {Writing MlPicTeX file \jobname.pic/\jobname.ml }\fi}
 
\def\@wrmlwrite#1{\let\thepage\relax
   \edef\@tempa{\write\@mlwritefile{\string #1}}\expandafter\endgroup\@tempa
   \if@nobreak \ifvmode\nobreak\fi\fi\@esphack}
  \def\mlwrite{\@bsphack\begingroup\@sanitize\@wrmlwrite}
\makeatother
% \mlDef{name}{text} defines a value (latexBox"name") of type MLgraph__picture
% which can be used as part of the resulting picture
\newcommand{\mlDef}[2]{% Declaration of horizontal boxes
    \advance\mlBoxCount by 1%
    \setbox\mlBoxCount=\hbox{#2}%
  \mlwrite%
  {(****************************************************************}%
  \mlwrite%
  { *  building box  : #1}%
  \mlwrite%
  { ****************************************************************)}%
    \mlwrite%
{add_latexBox (("#1",\the\mlBoxCount,"\the\wd\mlBoxCount","\the\ht\mlBoxCount","\the\dp\mlBoxCount"));;}}%%%%%%%%%%::!latexBoxTable;}}
\newcommand{\mlOpen}[1]{%
	\mlwrite{##open #1;;}%
%	\mlwrite{open #1;;}%
        \mlwrite{load_object #1;;}%
}%
\newcommand{\mlDir}[1]{%
	\mlwrite{\#}%
	\mlwrite{#1;;}%
}%
\newcommand{\mlDefGen}[3]{% Declaration of horizontal boxes
    \advance\mlBoxCount by 1%
    \setbox\mlBoxCount=\hbox{#3}%
    \mlwrite%
{latexBoxTable:=!latexBoxTable@[(latexPictureGen #2 ("#1",\the\mlBoxCount,"\the\wd\mlBoxCount","\the\ht\mlBoxCount","\the\dp\mlBoxCount"))];;}}%
\newcommand{\mlDefBox}[2]{% Inclusion of an existent box
    \mlwrite%
{latexBoxTable:=!latexBoxTable@[(latexPicture("#1",\the#2,"\the\wd#2","\the\ht#2","\the\dp#2"))];;}}%
\newcommand{\mlVDef}[2]{%  Declaration of vertical boxes
    \advance\mlBoxCount by 1%
    \setbox\mlBoxCount=\vbox{#2}%
    \mlwrite%
{latexBoxTable:=!latexBoxTable@[latexPicture("#1",\the\mlBoxCount,"\the\wd\mlBoxCount","\the\ht\mlBoxCount","\the\dp\mlBoxCount")];;}}%
% Expression of type MLgraph__picture that describes the picture
\newcommand{\mlBody}[2]{\mlToks={\jobname.pic/#1.tex}%
  \mlwrite{ }
  \mlwrite%
  {(****************************************************************}%
  \mlwrite%
  { * building picture #1 }
  \mlwrite%
  { ****************************************************************)}%
  \mlwrite{ }
%   \mlwrite{let latexBox s = fst(assoc s !latexBoxTable) }%
%   \mlwrite{and latexBoxes = map (fun (s,(p,_)) -> (s,p)) !latexBoxTable }%
%   \mlwrite{in }%
   \mlwrite{ let mlGraphPicture = }%
   \mlwrite{  #2 }%
   \mlwrite{  in makeLatexPicture mlGraphPicture !latexBoxTable "#1";; }%
}
% initializations for MLgraph pictures
\newenvironment{mlPic}{%
  \mlBoxCount=\the\count14% the box counter
   \mlwrite{(*--------- mlPic Description output ---------*)}%
   \mlwrite{ }%
  \mlwrite%
  {(****************************************************************}%
  \mlwrite%
  { * starting a new picture}%
  \mlwrite%
  { ****************************************************************)}%
  \mlwrite{ }%
   \mlwrite{empty_latexBoxTable ();;}%
   \mlwrite{ }%
 }{%
  \mlwrite{ }%
  \mlwrite{(*------End of mlPic description output ------*)}%
  \mlwrite{ }%
\mlReadFun{\the\mlToks}}
\newcommand{\mlReadFun}[1]{\openin\mlRead=#1
\ifeof\mlRead\closein\mlRead\message{#1\space is empty}%
\else{\closein\mlRead}\setlength{\mlLen}{\unitlength}%
\setlength{\unitlength}{1pt}\input{#1}%
\setlength{\unitlength}{\mlLen}\fi}
\newcommand{\mlPut}[3]{\put#1{\special{ps:gsave currentpoint currentpoint translate [ #2 0 0]concat neg exch neg exch translate}#3\special{ps:grestore}}}
\newcommand{\mlPutg}[4]{\put#1{\special{ps:gsave currentpoint currentpoint translate [ #2 0 0]concat neg
exch neg exch translate #3 setgray }#4\special{ps:grestore}}}
\newcommand{\mlPutrgb}[4]{\put#1{\special{ps:gsave (start) ==
[1 0 0 1 0 0 ] defaultmatrix ==
currentpoint [1 0 0 1 0 0] currentmatrix
initmatrix
currentpoint translate
[1 0 0 1 0 0] currentmatrix ==
#3
setrgbcolor  setmatrix moveto  currentpoint currentpoint translate [ #2 0 0]
concat neg exch neg exch translate  }#4\special{ps:grestore}}}
\newcommand{\mlPuthsb}[4]{\put#1{\special{ps:gsave currentpoint currentpoint translate [ #2 0 0]concat neg exch neg exch translate #3 sethsbcolor }#4\special{ps:grestore}}}
\newcommand{\mlPutgen}[4]{\put#1{\special{ps:gsave  currentpoint [1 0 0 1 0 0] currentmatrix initmatrix currentpoint translate #3 setmatrix moveto  currentpoint currentpoint translate [ #2 0 0]
concat neg exch neg exch translate}#4\special{ps:currentpoint grestore moveto}}}
\makeMlPicTex

%   \begin{mlPic}
%      \mlOpen{fa}
%      \mlOpen{fb}
%      \mlDir{da}
%      \mlDir{db}
%      \mlDef{a}{First LaTeX text}
%      \mlDef{b}{Second LaTeX text}
%      etc.
%      \mlBody{p}{ 
%          Your MLgraph picture which might use latexBox"a",latexBox"b",etc.
%          of type MLgraph__picture
%      }
%      \end{mlPic}
%
%Supposes that your LaTeX file uses the TeX box declaration conventios
\newcount{\mlBoxCount}\newtoks{\mlToks}\newread{\mlRead}\newlength{\mlLen}
\makeatletter
\def\makeMlPicTex{\if@filesw \newwrite\@mlwritefile
  \immediate\openout\@mlwritefile=\jobname.pic/\jobname.ml
  \mlwrite%
  {(****************************************************************}%
  \mlwrite%
  { *                                                               }%
  \mlwrite%
  { * File \jobname.pic/\jobname.ml created by  MLgraph v2.1 from \jobname.tex   }%
  \mlwrite%
  { *                                                               }%
  \mlwrite%
  { ****************************************************************)}%
  \mlwrite{ }%
  \mlwrite{include "mlgraph.ml";;}
  \mlwrite{picDir:="\jobname.pic";;}
  \mlwrite{set_latex_mode ();;}
%  \mlwrite{let addLatexBox lb = latexBoxTable:=!latexBoxTable@[lb];;}
  \mlwrite{ }%
  \typeout
  {Writing MlPicTeX file \jobname.pic/\jobname.ml }\fi}
 
\def\@wrmlwrite#1{\let\thepage\relax
   \edef\@tempa{\write\@mlwritefile{\string #1}}\expandafter\endgroup\@tempa
   \if@nobreak \ifvmode\nobreak\fi\fi\@esphack}
  \def\mlwrite{\@bsphack\begingroup\@sanitize\@wrmlwrite}
\makeatother
% \mlDef{name}{text} defines a value (latexBox"name") of type MLgraph__picture
% which can be used as part of the resulting picture
\newcommand{\mlDef}[2]{% Declaration of horizontal boxes
    \advance\mlBoxCount by 1%
    \setbox\mlBoxCount=\hbox{#2}%
  \mlwrite%
  {(****************************************************************}%
  \mlwrite%
  { *  building box  : #1}%
  \mlwrite%
  { ****************************************************************)}%
    \mlwrite%
{add_latexBox (("#1",\the\mlBoxCount,"\the\wd\mlBoxCount","\the\ht\mlBoxCount","\the\dp\mlBoxCount"));;}}%%%%%%%%%%::!latexBoxTable;}}
\newcommand{\mlOpen}[1]{%
	\mlwrite{##open #1;;}%
%	\mlwrite{open #1;;}%
        \mlwrite{load_object #1;;}%
}%
\newcommand{\mlDir}[1]{%
	\mlwrite{\#}%
	\mlwrite{#1;;}%
}%
\newcommand{\mlDefGen}[3]{% Declaration of horizontal boxes
    \advance\mlBoxCount by 1%
    \setbox\mlBoxCount=\hbox{#3}%
    \mlwrite%
{latexBoxTable:=!latexBoxTable@[(latexPictureGen #2 ("#1",\the\mlBoxCount,"\the\wd\mlBoxCount","\the\ht\mlBoxCount","\the\dp\mlBoxCount"))];;}}%
\newcommand{\mlDefBox}[2]{% Inclusion of an existent box
    \mlwrite%
{latexBoxTable:=!latexBoxTable@[(latexPicture("#1",\the#2,"\the\wd#2","\the\ht#2","\the\dp#2"))];;}}%
\newcommand{\mlVDef}[2]{%  Declaration of vertical boxes
    \advance\mlBoxCount by 1%
    \setbox\mlBoxCount=\vbox{#2}%
    \mlwrite%
{latexBoxTable:=!latexBoxTable@[latexPicture("#1",\the\mlBoxCount,"\the\wd\mlBoxCount","\the\ht\mlBoxCount","\the\dp\mlBoxCount")];;}}%
% Expression of type MLgraph__picture that describes the picture
\newcommand{\mlBody}[2]{\mlToks={\jobname.pic/#1.tex}%
  \mlwrite{ }
  \mlwrite%
  {(****************************************************************}%
  \mlwrite%
  { * building picture #1 }
  \mlwrite%
  { ****************************************************************)}%
  \mlwrite{ }
%   \mlwrite{let latexBox s = fst(assoc s !latexBoxTable) }%
%   \mlwrite{and latexBoxes = map (fun (s,(p,_)) -> (s,p)) !latexBoxTable }%
%   \mlwrite{in }%
   \mlwrite{ let mlGraphPicture = }%
   \mlwrite{  #2 }%
   \mlwrite{  in makeLatexPicture mlGraphPicture !latexBoxTable "#1";; }%
}
% initializations for MLgraph pictures
\newenvironment{mlPic}{%
  \mlBoxCount=\the\count14% the box counter
   \mlwrite{(*--------- mlPic Description output ---------*)}%
   \mlwrite{ }%
  \mlwrite%
  {(****************************************************************}%
  \mlwrite%
  { * starting a new picture}%
  \mlwrite%
  { ****************************************************************)}%
  \mlwrite{ }%
   \mlwrite{empty_latexBoxTable ();;}%
   \mlwrite{ }%
 }{%
  \mlwrite{ }%
  \mlwrite{(*------End of mlPic description output ------*)}%
  \mlwrite{ }%
\mlReadFun{\the\mlToks}}
\newcommand{\mlReadFun}[1]{\openin\mlRead=#1
\ifeof\mlRead\closein\mlRead\message{#1\space is empty}%
\else{\closein\mlRead}\setlength{\mlLen}{\unitlength}%
\setlength{\unitlength}{1pt}\input{#1}%
\setlength{\unitlength}{\mlLen}\fi}
\newcommand{\mlPut}[3]{\put#1{\special{ps:gsave currentpoint currentpoint translate [ #2 0 0]concat neg exch neg exch translate}#3\special{ps:grestore}}}
\newcommand{\mlPutg}[4]{\put#1{\special{ps:gsave currentpoint currentpoint translate [ #2 0 0]concat neg
exch neg exch translate #3 setgray }#4\special{ps:grestore}}}
\newcommand{\mlPutrgb}[4]{\put#1{\special{ps:gsave (start) ==
[1 0 0 1 0 0 ] defaultmatrix ==
currentpoint [1 0 0 1 0 0] currentmatrix
initmatrix
currentpoint translate
[1 0 0 1 0 0] currentmatrix ==
#3
setrgbcolor  setmatrix moveto  currentpoint currentpoint translate [ #2 0 0]
concat neg exch neg exch translate  }#4\special{ps:grestore}}}
\newcommand{\mlPuthsb}[4]{\put#1{\special{ps:gsave currentpoint currentpoint translate [ #2 0 0]concat neg exch neg exch translate #3 sethsbcolor }#4\special{ps:grestore}}}
\newcommand{\mlPutgen}[4]{\put#1{\special{ps:gsave  currentpoint [1 0 0 1 0 0] currentmatrix initmatrix currentpoint translate #3 setmatrix moveto  currentpoint currentpoint translate [ #2 0 0]
concat neg exch neg exch translate}#4\special{ps:currentpoint grestore moveto}}}
\makeMlPicTex
